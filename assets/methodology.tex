\subsection{Data Acquisition and Preprocessing}
The dataset for this project is constructed from publicly available NBA game data, sourced from the nba\_api. The dataset will encompass all 105 games (82 regular season, 23 playoff) played by the Indiana Pacers during the 2024-2025 season.

The following data were collected for each of these games:
\begin{itemize}
    \item Game-level data: Date, opponent, location (home/away), and final score.
    \item Player-level data: Detailed box scores for every player who participated in each game (for both the Pacers and their opponent). This includes minutes played (MP) and standard counting stats (points, rebounds, assists, steals, blocks, turnovers, fouls, FGA, FGM, 3PA, 3PM, FTA, FTM).
\end{itemize}
The raw data was preprocessed into a structured format where each row represents a single game $(N=105)$. The primary target variable will be a binary outcome: Pacers\_Win (1 for a win, 0 for a loss). All other collected data was used to engineer predictive features. Preprocessing also involved handling missing values, such as removing players who were on the roster but did not play (DNP).

\subsection{Feature Engineering: Player Archetypes}

A central component of our methodology is to move beyond individual player statistics and model the composition of the lineups. As proposed in the abstract, we operationalized this by creating ``player archetypes.''

\begin{itemize}
    \item Archetype Clustering: We applied a K-Means clustering algorithm to define the player archetypes. The clustering was performed on a normalized dataset of all players (from both the Pacers and their opponents) who played significant minutes during the season. The features used for clustering included per-36-minute standardized statistics and advanced metrics (e.g., True Shooting Percentage (TS\%), Usage Rate (USG\%), Assist Rate, Rebound Rate) to capture a player's style and on-court role.
    \item Archetype Labeling: After clustering, we manually inspected the statistical profile of each cluster to assign a descriptive label.
    \item Game-level Feature Vector: For each of the 105 games, we created a feature vector representing the lineup composition. This vector is a count of how many players belonging to each archetype were among the top five minute-getters for that game. This was done for both the Pacers and their opponent.
\end{itemize}

The final feature set for each game will include:

\begin{itemize}
    \item Pacers\_Archetype\_1\_Count
    \item Pacers\_Archetype\_2\_count
    \item \ldots
    \item Opponent\_Archetype\_1\_Count
    \item Opponent\_Archetype\_2\_Count
    \item \ldots
\end{itemize}

This approach directly models the interaction of lineup archetypes, addressing the hypothesis that the type of opponent impacts the Pacers' win probability.

\subsection{Model Development and Evaluation}

Given the binary nature of our target variable (Pacers\_Win), this project formulates the problem as a binary classification task. Due to the small dataset size $(N=105)$, we prioritized models that are less prone to overfitting and offer high interpretability.

\subsubsection{Selected Models}
We implemented and compared several classification algorithms:
\begin{itemize}
    \item \textbf{Logistic Regression}: A baseline model that is highly interpretable, allowing us to quantify how the presence of specific Pacers or opponent archetypes directly impacts the log-odds of winning.
    \item \textbf{K-Nearest Neighbors (K-NN)}: This model works by finding the `$k$' most similar games from the past (the ``nearest neighbors'') and then predicting the outcome based on how those similar games turned out.
    \item \textbf{Naive Bayes}: The Naive Bayes probabilistic classifier calculates the likelihood of a win or loss given the lineup archetypes.
\end{itemize}
\subsubsection{Model Validation}
We employed a 80/20 train-test split where the games were selected at random so that 80\% were used to train and 20\% were used to the test the model's ability to predict the outcome of the game.

\subsubsection{Evaluation Metrics}
Model performance is assessed using a suite of standard classification metrics:
\begin{itemize}
    \item \textbf{Accuracy}: The overall percentage of correctly predicted game outcomes.
    \item \textbf{Precision, Recall, and F1-Score}: These metrics are especially crucial if the win-loss record is imbalanced (e.g., if the Pacers had a 70-win season, a model that always predicts ``Win'' would have high accuracy but zero utility).
\end{itemize}

The deliverable is a model that is reasonably accurate and may provide actionable insights into which lineup compositions and opponent archetypes led to wins and losses for the 2024-2025 Pacers. Although the model is not highly accurate--as you will see in the results--the model shows does show strong potential for meaningfully predicting the outcomes based on the player archetypes if it is supplied with more game data.