\subsection{Player Archetype Analysis}
We applied K-Means clustering to identify different player archetypes based on playing style and player statistics. The analysis included all players, Pacers and opponents, who logged more than 10 minutes during the 2024-2025 season in regular and postseason games. The stats of the five player archetypes identified through clustering are shown in figure 1.

\begin{itemize}
    \item \textbf{Archetype 0}: These players contribute minimally across the statistical categories not assuming a primary offensive or defensive role. These players have a low average scoring output and do not have a large number of field goals attempted per 36 minutes. They provide moderate rebounding and assist support, while having a low turnover rate. These players typically function as role players within a team's structure, providing support when needed.
    \item \textbf{Archetype 1}: These players are the disruptive defenders. With an average of 4 steals per 36 minutes these players thrive on forcing turnovers and turning defense into offense. The rest of their 36-minute stats are relatively high, making them a versatile guards or wings who are able to contribute to the team with hustle and defensive intensity.
    \item \textbf{Archetype 2}: These players are characterized by elite defensive metrics in the paint. Averaging 10 rebounds are nearly 3 blocks per 36 minutes, these players anchor a team defense and control the paint. These players have a limited offensive role, most likely finishing plays rather than initiating plays, through assists. These players align with the traditional defensive big men, whose primary value lies in rim protection and being able to secure possessions through rebounding. 
    \item \textbf{Archetype 3}: These players combine high scoring with elite play making ability. Averaging nearly 20 points and over 8 assists per 36 minutes, these players function as a team's offensive engine. The elevated turnover rate for this archetype reflects that these players have a central role in ball handling and decision making for a team's offense. These players best reflect a star point guard or primary ball handler that is essential to a team's success. 
    \item \textbf{Archetype 4}: These players are scoring machines. Averaging 25 points and nearly 4 three pointers made per 36 minutes, these players are the primary scorers for a team's offense. This is also shown in the fact that these players have the highest field goal attempts per 36 minutes. These players contribute fine in rebounding and assists but have a primary function of offensive scoring. These players align with scoring specialists whose value to a team comes from their ability to generate points at high efficiency and volume.
\end{itemize}

\begin{figure}
    \centering
    \centerline{\includegraphics[width=1\linewidth]{assets/B365_Final_Report_PlayerClusters.png}}
    \caption{Player Archetype per 36 Minutes Stats}
    \label{fig:1}
\end{figure}
\subsection{Model Evaluation}
Three classification algorithms, K-Nearest Neighbors (KNN), Naïve Bayes, and Logistic Regression, were evaluated on the dataset. Their performance is shown in Figure 2
\begin{itemize}
    \item \textbf{KNN}: achieved the highest overall accuracy 0.62 but failed to correctly classify any losses. KNN had a strong performance for classifying wins with precision being 0.68, recall being 0.87, and F1-score being 0.76.
    \item \textbf{Naive Bayes}: obtained an accuracy of 0.57. The Naïve Bayes algorithm did demonstrate a slightly better time at classifying losses with precision being 0.20, recall being 0.17, and F1-score being 0.18. The Naïve Bayes algorithm like the KNN algorithm stayed strong at classifying wins with precision being 0.69, recall being 0.73 and F1-score being 0.71.
    \item \textbf{Logistic Regression}: had the lowest accuracy of 0.52 and also did a slightly better time at classifying losses compared to KNN with precision being 0.17, recall being 0.17, and F1-score being 0.17. Logistic Regression like the other algorithms performed wall at classifying wins with precision being 0.67, recall being 0.67, and F1-score being 0.67. 
\end{itemize}
\begin{figure}
    \centering
    \includegraphics[width=1\linewidth]{assets/B365_Final_Report_ModelComparison.png}
    \caption{Model Performance Comparison}
    \label{fig:2}
\end{figure}



